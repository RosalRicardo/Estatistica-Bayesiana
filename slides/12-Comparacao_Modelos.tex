\section{Comparação de Modelos}

\subsection{Leituras Recomendadas}
\begin{frame}{Comparação de Modelos - Leituras Recomendadas}
    \begin{vfilleditems}
        \item \textcite{gelman2013bayesian} - Capítulo 7: Evaluating, comparing, and expanding models
        \item \textcite{gelman2020regression} - Capítulo 11, Seção 11.8: Cross validation
        \item \textcite{mcelreath2020statistical} - Capítulo 7, Seção 7.5: Modelcomparison
        \item \textcite{vehtariPracticalBayesianModel2015}
        \item Tutorial do \texttt{loo} de \textcite{loo}
        \item \textcite{storopoli2021estatisticabayesianaR} - Comparação de Modelos
        \item \textcite{spiegelhalter2002bayesian}
        \item \textcite{van2005dic}
        \item \textcite{watanabe2010asymptotic}
        \item \textcite{gelfand1996model}
        \item \textcite{watanabe2010asymptotic}
        \item \textcite{geisser1979predictive}
    \end{vfilleditems}
\end{frame}

\subsection{Por quê Comparar Modelos?}
\begin{frame}{Por quê Comparar Modelos?}
    Depois de estimarmos um modelo Bayesiano,
    muitas vezes queremos medir sua precisão preditiva, por si só ou para
    fins de comparação, seleção ou cálculo de média do modelo \parencite{geisser1979predictive}.
\end{frame}

\begin{frame}{Mas, e as Verificações Preditivas da Posterior?}
    É uma maneira subjetiva e arbitrária de analisarmos e compararmos modelos
    entre si usando sua precisão preditiva.
    \vfill
    Há uma maneira objetiva de compararmos modelos Bayesianos com uma
    métrica robusta que nos ajude a selecionar qual o melhor modelo dentre o rol
    de modelos candidatos.
    \vfill
    Ter uma maneira objetiva de comparar modelos e escolher o melhor dentre eles
    é muito importante pois no \textit{workflow} Bayesiano geralmente temos diversas
    iterações entre \textit{prioris} e funções de verossimilhança o que ocasiona na
    criação de diversos modelos diferentes \parencite{gelmanBayesianWorkflow2020}.
\end{frame}

\subsection{Técnicas de Comparação de Modelos}
\begin{frame}{Técnicas de Comparação de Modelos}
    Temos diversas técnicas de comparação de modelos que usam a precisão preditiva,
    sendo as principais:
    \begin{vfilleditems}
        \item \textit{Leave-one-out cross-validation} (LOO)
        \parencite{vehtariPracticalBayesianModel2015}
        \item \textit{Deviance Information Criterion} (DIC)
        \parencite{spiegelhalter2002bayesian},  mas sabe-se que tem alguns problemas,
        que surgem em parte por não ser totalmente Bayesiano,
        pois se baseia em uma estimativa pontual \parencite{van2005dic}
        \item \textit{Widely Applicable Information Criteria} (WAIC)
        \parencite{watanabe2010asymptotic}, totalmente Bayesiano no sentido
        de que usa toda a distribuição posterior, e é assintoticamente igual ao
        LOO \parencite{vehtariPracticalBayesianModel2015}
    \end{vfilleditems}
\end{frame}

\begin{frame}{Interlúdio Histórico}
    \small
    Antigamente, não havia esse poder computacional e abundância de dados.
    Comparação de modelos eram baseados em uma métrica de divergência teórica
    oriunda da entropia da teoria da informação:
    $$
    H(p) = - \operatorname{E}\log(p_i) = -\sum^N_{i=1} p_i \log(p_i)
    $$
    \small
    Calculamos a divergência\footnote{\textit{divergence}} multiplicando por
    $-2$\footnote{razões históricas},
    então menores valores são melhores:
    $$
    D(y, \boldsymbol{\theta}) = -2 \cdot \underbrace{\sum^N_{i=1} \log \frac{1}{S}\sum^S_{s=1} P(y_i \mid \boldsymbol{\theta}^s)}_{\text{\textit{log pointwise predictive density} -- lppd}}
    $$
    \footnotesize
    onde $N$ é o tamanho da amostra e $S$ é o número de amostras simuladas da posterior.
\end{frame}

\begin{frame}{Interlúdio Histórico -- AIC \parencite{akaike1998information}}
    $$\text{AIC} = D(y, \boldsymbol{\theta}) + 2k = -2 \text{lppd}_{\text{mle}} + 2k$$
    onde $k$ é o número de parâmetros livres do modelo e $\text{lppd}_{\text{mle}}$ é
    a estimação de máxima verossimilhança (MLE) da lppd.
    \vfill
    AIC é uma aproximação que somente pode ser confiável quando:
    \begin{vfilleditems}
        \item As \textit{prioris} são uniformes (\textit{flat priors}) ou dominadas totalmente pela função de verossimilhança
        \item A posterior é aproximadamente uma distribuição Gaussiana/normal multivariada
        \item O tamanho da amostra $N$ é muito maior que número de parâmetros livres $k$: $N \gg k$
    \end{vfilleditems}
\end{frame}

\begin{frame}{Interlúdio Histórico -- DIC \parencite{spiegelhalter2002bayesian}}
    Uma generalização do AIC, onde substituímos a estimação de máxima verossimilhança
    pela média da posterior e $k$ por uma correção de viés baseada nos dados:
    $$
    DIC = D(y, \boldsymbol{\theta}) + k_{\text{DIC}} = -2 \text{lppd}_{\text{Bayes}}
    +2 \underbrace{\left( \text{lppd}_{\text{Bayes}} - \frac{1}{S} \sum^S_{s=1} \log P(y \mid \boldsymbol{\theta}^s) \right)}_{\text{$k$ corrigido de viés}}
    $$
    DIC remove a restrição das \textit{prioris} uniformes de AIC, mas mesmo assim
    mantém os presupostos da posterior ser uma distribuição Gaussiana/normal multivariada
    e que $N \gg k$
\end{frame}

\subsubsection{Precisão Preditiva}
\begin{frame}{Precisão Preditiva}
    Com o poder computacional que temos hoje não precisamos de aproximações\footnote{AIC, DIC etc.}.
    \vfill
    Podemos discutir métricas objetivas de \textbf{precisão preditiva}.
    \vfill
    Mas antes vamos definir o que é precisão preditiva.
\end{frame}

\begin{frame}{Precisão Preditiva}
    \begin{defn}[Precisão Preditiva]
        Bayesianos mensuram precisão preditiva usando simulações da distribuição posterior
        $\tilde{y}$ do modelo. Para isso temos a distribuição preditiva posterior
        (\textit{predictive posterior distribution}):
        $$
        p(\tilde{y} \mid y) = \int p(\tilde{y}_i \mid \theta) p(\theta \mid y) d \theta
        $$
        \small
        Onde $p(\theta \mid y)$ é a distribuição posterior do modelo\footnote{aquela
        que o \texttt{rstanarm} e \texttt{brms} estima para nós}.
        A fórmula acima significa que calculamos a integral de toda a
        probabilidade conjunta da distribuição posterior preditiva com a
        distribuição posterior do nosso modelo
        \normalsize
        Quanto \textbf{maior} a distribuição preditiva posterior
        $p(\tilde{y} \mid y)$ \textbf{melhor} será a precisão preditiva do modelo.
    \end{defn}
\end{frame}

\begin{frame}{Precisão Preditiva}
    Para mantermos comparabilidade entre amostras, calculamos a
    esperança dessa medida\footnote{do inglês \textit{expectation} que pode ser
    também interpretada como a média ponderada} para cada uma das $N$ observações
    da amostra:

    $$
    \operatorname{elpd} = \sum_{i=1}^N \int p_t(\tilde{y}_i) \log p(\tilde{y}_i \mid y) d \tilde{y}
    $$

    onde $\operatorname{elpd}$ é esperança do log da densidade preditiva pontual
    (\textit{expected log pointwise predictive density}) e
    $p_t(\tilde{y}_i)$ é a distribuição representando o verdadeiro processo
    generativo dos dados para $\tilde{y}_i$.
    Os $p_t(\tilde{y}_i)$ são desconhecidos e geralmente usamos validação
    cruzada\footnote{\textit{Cross Validation}} ou aproximação para
    a estimação da $\operatorname{elpd}$.
\end{frame}

\subsubsection{\textit{Leave-One-Out Cross-Validation} (LOO)}
\begin{frame}{\textit{Leave-One-Out Cross-Validation} (LOO)}
    Podemos calcular a $\operatorname{elpd}$ usando LOO
    \parencite{vehtariPracticalBayesianModel2015}:
    $$
    \operatorname{elpd}_{\text{loo}} = \sum_{i=1}^N \log p(y_i \mid y_{-i})
    $$
    onde
    $$
    p(y_i \mid y_{-i}) = \int p(y_i \mid \theta) p(\theta \mid y_{-i}) d \theta
    $$
    que é a densidade preditiva com uma observação a menos condicionada nos
    dados sem a observação $i$ ($y_{-i}$). Quase sempre usamos a aproximação
    PSIS-LOO\footnote{mais sobre isso já já...} pela sua robustez e baixo custo
    computacional.
\end{frame}

\subsubsection{\textit{Widely Applicable Information Criteria} (WAIC)}
\begin{frame}{\textit{Widely Applicable Information Criteria} (WAIC)}
    \footnotesize
    WAIC \parencite{watanabe2010asymptotic}, assim como o LOO também é uma
    abordagem alternativa para calcularmos a $\operatorname{elpd}$ e é definida como:

    $$
    \widehat{\operatorname{elpd}}_{\text{waic}} = \widehat{\operatorname{lppd}} - \widehat{p}_{\text{waic}}
    $$

    onde $\widehat{p}_{\text{waic}}$ é o número estimado efetivo de paramêtros e
    calculado com base em:

    $$
    \widehat{p}_{\text{waic}} = \sum_{i=1}^N \operatorname{var}_{\text{post}} (\log p(y_i \mid \theta))
    $$

    que conseguimos calcular usando a variância posterior do log da densidade preditiva para cada observação $y_i$:

    $$
    \widehat{p}_{\text{waic}} = \sum_{i=1}^N V^S_{s=1} (\log p(y_i \mid \theta^s))
    $$
    onde $V^S_{s=1}$ representa a variância da amostra:

    $$
    V^S_{s=1} a_s = \frac{1}{S-1} \sum^S_{s=1} (a_s - \bar{a})^2
    $$
\end{frame}

\subsubsection{\textit{K-fold Cross-Validation} (K-fold CV)}
\begin{frame}{\textit{K-fold Cross-Validation} (K-fold CV)}
    Da mesma maneira que conseguimos cacular a $\operatorname{elpd}$ usando LOO
    com $N-1$ partições da amostra podemos também calcular com qualquer número de
    partições que quisermos.
    \vfill
    Tal abordagem é chamada de \textbf{validação cruzada usando $K$ partições}
    (\textit{$K$-fold Cross-Validation}, encurtado para \textit{$K$-fold CV}).
    \vfill
    Ao contrário de LOO, não conseguimos aproximar a
    $\operatorname{elpd}$ usando \textit{$K$-fold CV} e precisamos fazer a computação
    atual da $\operatorname{elpd}$ sobre $K$ partições que quase sempre envolve
    um \textbf{alto custo computacional}.
\end{frame}

\subsection{\textit{Pareto Smoothed Importance Sampling} LOO (PSIS-LOO)}
\begin{frame}{\textit{Pareto Smoothed Importance Sampling} LOO (PSIS-LOO)}
    O PSIS usa \textbf{amostragem de importância}\footnote{\textit{importance sampling}},
    o que significa apenas que usa a abordagem de pesos de importância.
    \vfill
    A \textbf{suavização de Pareto} é uma técnica para tornar os pesos de importância
    mais confiáveis.
\end{frame}
\begin{frame}{Amostragem de Importância (\textit{Importance Sampling})}
    Se as $N$ amostras são condicionalmente independentes\footnote{ou seja
    são independentes condicionadas aos parâmetros do modelo, que é o pressuposto
    básico de qualquer modelo probabilístico Bayesiano} \parencite{gelfand1992model}
    podemos avaliar LOO com amostrad $\boldsymbol{\theta}^s$ da posterior
    $P(\theta \mid y)$ usando \textbf{pesos de importância}:
    $$
    r_i^s=\frac{1}{P(y_i|\theta^s)} \propto \frac{P(\theta^s|y_{-i})}{P(\theta^s|y)}
    $$
    Para então conseguirmos \textit{Importance Sampling Leave-One-Out} (IS-LOO):
    $$
    P(\tilde{y}_i|y_{-i})
    \approx
    \frac{\sum_{s=1}^S r_i^s P(\tilde{y}_i|\theta^s)}{\sum_{s=1}^S r_i^s}
    $$
\end{frame}

\begin{frame}{Amostragem de Importância (\textit{Importance Sampling})}
    Porém a posterior $P(\theta \mid y)$ geralmente possui baixa variância e caudas
    mais curtas que as distribuições LOO $P(\theta \mid y_{-1})$ então se usarmos:
    $$
    P(\tilde{y}_i|y_{-i}) \approx \frac{\sum_{s=1}^S r_i^s P(\tilde{y}_i|\theta^s)}{\sum_{s=1}^S r_i^s}
    $$
    podemos gerar instabilidades pois os $r_i$ podem ter variância alta ou até infinita.
\end{frame}

\begin{frame}{Amostragem de Importância com Suavização de Pareto \textit{Pareto Smoothed Importance Sampling}}
    Podemos aprimorar a estimativa IS-LOO usando uma \textbf{suavização de Pareto}
    (\textit{Pareto Smoothed Importance Sampling}) \parencite{vehtariPracticalBayesianModel2015}
    \vfill
    Quando a cauda da distribuição dos pesos de importância é longa, um uso direto
    da amostragem de importância é sensível a um ou alguns valores grandes.
    Ajustando uma distribuição de Pareto generalizada à cauda superior dos pesos de
    importância, suavizamos esses valores.
\end{frame}
\begin{frame}{\textit{Pareto Smoothed Importance Sampling} LOO (PSIS-LOO)}
    Por fim temos PSIS-LOO:
    $$
    \widehat{\operatorname{elpd}}_{\rm psis-loo} =
    \sum_{i=1}^n \log
    \left(\frac{\sum_{s=1}^S w_i^s P(y_i|\theta^s)}{\sum_{s=1}^Sw_i^s} \right)
    $$
    onde $w$ é o peso truncado.
\end{frame}

\begin{frame}{\textit{Pareto Smoothed Importance Sampling} LOO (PSIS-LOO)}
    \small
    Usamos o parâmetro estimado de forma $\widehat{k}$ da distribuição de Pareto
    dos pesos de importância para avaliar a confiabilidade da estimativa:
    \begin{vfilleditems}
        \item \small $k < \frac{1}{2}$ a variância dos pesos de importância é finita,
        o teorema do limite central se mantém, e a estimativa converge rapidamente
        \item \small $\frac{1}{2} < k < 1$ a variância dos pesos de importância é infinitas,
        mas a média existe, o teorema do limite central generalizado
        para distribuições estáveis se mantém, e a convergência da estimativa é mais
        lenta
        A variação da estimativa PSIS é finita, mas pode ser grande.
        \item \small $k > 1$ a variância e a média da distribuição de pesos de importância não
        existe. A variação da estimativa PSIS é finita, mas pode ser grande
    \end{vfilleditems}
    \vfill
    \small
    Qualquer valor de $\widehat{k}$ maior que $0.5$ é sinal de alerta mas na prática
    ainda há um bom desempenho com $\widehat{k}$ até $0.7$
\end{frame}

\subsection{Comparação de Modelos no \texttt{rstarnarm}}
\begin{frame}[fragile]{Comparação de Modelos no \href{http://mc-stan.org/rstanarm/}{\texttt{rstanarm}}}
    \begin{lstlisting}
    library(loo)

    loo_1 <- loo(rstanarm_model_1)
    loo_2 <- loo(rstanarm_model_2)
    loo_3 <- loo(rstanarm_model_3)

    loo_compare(loo_1, loo_2, loo_3)
    \end{lstlisting}
\end{frame}

\subsection{Comparação de Modelos no \texttt{brms}}
\begin{frame}[fragile]{Comparação de Modelos no \href{https://paul-buerkner.github.io/brms/}{\texttt{brms}}}
    \begin{lstlisting}
    library(loo)

    loo_1 <- loo(brms_model_1)
    loo_2 <- loo(brms_model_2)
    loo_3 <- loo(brms_model_3)

    loo_compare(loo_1, loo_2, loo_3)
    \end{lstlisting}
\end{frame}
