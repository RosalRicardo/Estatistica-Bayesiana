\section{\textit{Prioris}}

\subsection{Leituras Recomendadas}
\begin{frame}{\textit{Prioris} - Leituras Recomendadas}
    \begin{vfilleditems}
        \item \textcite{gelman2013bayesian}:
        \begin{vfilleditems}
            \item Capítulo 2: Single-parameter models
            \item Capítulo 3: Introduction to multiparameter models
        \end{vfilleditems}
        \item \textcite{mcelreath2020statistical} - Capítulo 4: Geocentric Models
        \item \textcite{gelman2020regression}:
        \begin{vfilleditems}
            \item Capítulo 9, Seção 9.3: Prior information and Bayesian synthesis
            \item Capítulo 9, Seção 9.5: Uniform, weakly informative, and informative priors in regression
        \end{vfilleditems}
        \item \textcite{vandeschootBayesianStatisticsModelling2021}
        \item \textcite{storopoli2021estatisticabayesianaR} - Priors
        \item \href{http://mc-stan.org/rstanarm/articles/priors.html}{Vinheta do \texttt{rstanarm} sobre \textit{prioris}}
        \item \href{https://paul-buerkner.github.io/brms/reference/set_prior.html}{Documentação sobre \textit{prioris} do \texttt{brms}}
    \end{vfilleditems}
\end{frame}

\begin{frame}{Probabilidade \textit{Priori}}
    A Estatística Bayesiana é caracterizada pelo uso de informação prévia
    embutida como probabilidade prévia $P(\theta)$, chamada de \textit{priori}:
    $$
    \underbrace{P(\theta \mid y)}_{\textit{Posterior}} = \frac{\overbrace{P(y \mid  \theta)}^{\text{Verossimilhança}} \cdot \overbrace{P(\theta)}^{\textit{Priori}}}{\underbrace{P(y)}_{\text{Constante Normalizadora}}}
    $$
\end{frame}

\subsection{A subjetividade da \textit{Priori}}
\begin{frame}{A subjetividade da \textit{Priori}}
    \begin{vfilleditems}
        \item Muitas críticas à estatística Bayesiana, se dá pela subjetividade da
        elucidação da probabilidade *a priori* de certas hipóteses ou parâmetros de
        modelos.
        \item A subjetividade é algo indesejado na idealização do cientista e do
        método científico.
        \item Tudo que envolve ação humana nunca será 100\% objetivo.
        Temos subjetividade em tudo e ciência \textcolor{red}{não} é um exceção.
        \item O próprio processo dedutivo e criativo de formulação de teoria e
        hipóteses não é algo objetivo.
        \item A estatística frequentista, que bane o uso de probabilidades \textit{a priori}
        também é subjetiva, pois há \textbf{MUITA} subjetividade em especificar um modelo
        e uma função de verossimilhança \parencite{jaynesProbabilityTheoryLogic2003, vandeschootBayesianStatisticsModelling2021}
    \end{vfilleditems}
\end{frame}

\begin{frame}{Como Incorporar Subjetividade}
    \begin{vfilleditems}
        \item A estatística Bayesiana \textbf{abraça} a subjetividade enquanto a
        estatística frequentista a \textbf{proíbe}.
        \item Para a estatística Bayesiana, \textbf{subjetividade guiam nossas inferências e nos levam a modelos mais robustos},
        confiáveis e que podem auxiliar à tomada de decisão.
        \item Já para a estatística frequentista, \textbf{subjetividade é um tabu e todas inferências devem ser objetivas},
        mesmo que isso resulte em \textbf{esconder pressupostos dos modelos embaixo dos panos}.
        \item Estatística Bayesiana possui também pressupostos e subjetividade,
        mas estes são \textbf{enunciados e formalizados}.
    \end{vfilleditems}
\end{frame}

\subsection{Tipos de \textit{Prioris}}

\begin{frame}{Tipos de \textit{Prioris}}
    De maneira geral, podemos ter 3 tipos de \textit{priori} em uma abordagem
    Bayesiana \parencite{gelman2013bayesian, mcelreath2020statistical, vandeschootBayesianStatisticsModelling2021}:
    \begin{vfilleditems}
        \item uniforme (\textit{flat}): não recomendada
        \item fracamente informativa (\textit{weakly informative}): pequena restrição
        com um pouco de senso comum e baixo conhecimento de domínio incorporado
        \item informativa (\textit{informative}): conhecimento de domínio incorporado
    \end{vfilleditems}
\end{frame}

\subsubsection{\textit{Priori} Uniforme (\textit{Flat})}
\begin{frame}{\textit{Priori} Uniforme (\textit{Flat})}
    Parte da premissa que "tudo é possível". Não há limites na crença de que tamanho
    o valor deve ser ou quaisquer restrições.
    \vfill
    \textit{Prioris} uniformes e super-vagas geralmente não são recomendadas e algum
    esforço deve ser incluído para ter, pelo menos, \textit{prioris} um pouco
    informativas.
    \vfill
    Formalmente uma \textit{priori} uniforme é uma distribuição uniforme em todo o
    suporte possível do valor do parâmetros
    \begin{vfilleditems}
        \item parâmetros a serem estimados: $\{\theta \in \mathbb{R} : -\infty < \theta < \infty\}$
        \item resíduos ou erros do modelo: $\{\sigma \in \mathbb{R}^+ : 0 \leq \theta < \infty\}$
    \end{vfilleditems}
\end{frame}

\subsubsection{\textit{Priori} Fracamente Informativa (\textit{Weekly Informative})}
\begin{frame}{\textit{Priori} Fracamente Informativa (\textit{Weekly Informative})}
    Aqui começamos a entrar num palpite informado sobre o valor do parâmetro.
    Portanto não partimos da premissa que "tudo é possível".
    \vfill
    Recomendo sempre traduzir as \textit{prioris} do seu problema em algo
    centrado em $0$ e com desvio padrão $1$\footnote{isso chama-se normalização,
    transformar todas variáveis para terem média $0$ e desvio padrão $1$}:
    \vfill
    \begin{vfilleditems}
        \item $\theta \sim \text{Normal}(0, 1)$ (preferida do Andrew Gelman\footnote{Veja mais sobre a escolha de \textit{prioris} nessa \href{https://github.com/stan-dev/stan/wiki/Prior-Choice-Recommendations}{wiki do GitHub do \texttt{Stan}}})
        \item $\theta \sim \text{Student}(\nu=3, 0, 1)$ (preferida do Aki Vehtari)
        \item $\sigma \sim \text{Exponencial}(10)$
    \end{vfilleditems}
\end{frame}

\begin{frame}{Um exemplo de uma \textit{Priori} robusta}
    \footnotesize
    Um exemplo interessante vem de uma aula do Ben
    Goodrich\footnote{\url{https://youtu.be/p6cyRBWahRA}, caso queira ver o vídeo na íntegra, a parte que nos interessa de \textit{prioris} começa a partir do minuto 40}
    professor de Columbia e membro do grupo de pesquisa de \texttt{Stan}.
    \vfill
    Aqui ele fala sobre um dos maiores efeitos observados nas ciências sociais.
    Nas pesquisas de intenção de voto à eleição presidencial dos EUA de 2008 (Obama vs McCain),
    havia um apoio de quase 40\% do Obama de maneira geral.
    Se você trocasse a raça de um respondente de \lstinline!race_black = 0!
    para \lstinline!race_black = 1! isso gerava um aumento de aproximadamente 60\%
    na probabilidade do respondente votar no Obama\footnote{ele clama que isso é provavelmente
    o maior efeito encontrado na história das ciências sociais}.
    \vfill
    Em escala logit esses 60\% se
    traduziriam em um modelo binomial como um coeficiente
    $\beta_{\text{Race Black}} = 3.64$\footnote{$\log(\text{chance}) = \frac{e^{0.6}}{1 + e^{0.6}} \approx 3.644$}.
    Esse tamanho de efeito seria facilmente inferido com uma \textit{priori} $$\beta_{\text{Race Black}} \sim \text{Normal}(0, 1)$$
\end{frame}

\subsubsection{\textit{Priori} Informativa (\textit{Informative})}
\begin{frame}{\textit{Priori} Informativa (\textit{Informative})}
    Em alguns cenários é interessante usarmos uma \textit{priori} informativa.
    Geralmante são cenários que os dados são raros ou custosos de serem obtidos
    e que temos já algum conhecimento prévio sobre o fenômeno:
    \begin{vfilleditems}
        \item $\text{Normal}(5, 20)$
        \item $\text{Log-Normal}(0, 5)$
        \item $\text{Beta}(100, 9803)$\footnote{esta é usada nos modelos de COVID do grupo de pesquisa \href{https://codatmo.github.io}{CoDatMo} \texttt{Stan}}
    \end{vfilleditems}
\end{frame}

\subsection{\textit{Prioris} do \texttt{rstanarm} e \texttt{brms}}
\begin{frame}{\textit{Prioris} do \href{http://mc-stan.org/rstanarm/}{\texttt{rstanarm}} e \href{https://paul-buerkner.github.io/brms/}{\texttt{brms}}}
    Tanto \href{http://mc-stan.org/rstanarm/}{\texttt{rstanarm}}
    quanto \href{https://paul-buerkner.github.io/brms/}{\texttt{brms}} possui \textit{prioris}
    padrões incorporadas nos seus modelos.
    \vfill
    \textbf{Recomendo fortemente que você use uma \textit{priori} específica e não se atenha
    às prioris padrões do \texttt{rstanarm} ou \texttt{brms}}.
    \vfill
    \begin{vfilleditems}
        \item Elas refletem o estado-da-arte e as melhores práticas
        de especificação de \textit{prioris}
        \item Podem mudar conforme versões dos pacotes
        \item Não faz o seu modelo/código ser \textbf{replicável}
    \end{vfilleditems}
\end{frame}

\subsubsection{\textit{Prioris} do \texttt{rstanarm}}
\begin{frame}{\textit{Prioris} do \href{http://mc-stan.org/rstanarm/}{\texttt{rstanarm}}}
    \footnotesize
    \begin{tabular}{|l|p{.3\textwidth}|p{.3\textwidth}|}
        \toprule
        \textbf{Argumento}         & \textbf{Usado em}                                                                    & \textbf{Aplica-se à}                                                                                                     \\ \midrule
        \texttt{prior\_intercept}  & Todas funções de modelagem exceto \texttt{stan\_polr} e \texttt{stan\_nlmer}         & Constante (\textit{intercept}) do modelo, após centralização dos preditores                                              \\ \midrule
        \texttt{prior}             & Todas funções de modelagem                                                           & Coeficientes de Regressão, não inclui coeficientes que variam por grupo em modelos multiníveis (veja prior\_covariance)  \\ \midrule
        \texttt{prior\_aux}        & \texttt{stan\_glm}, \texttt{stan\_glmer}, \texttt{stan\_gamm4}, \texttt{stan\_nlmer} & Parâmetro auxiliar (ex: desvio padrão (standard error – DP), interpretação depende do modelo                             \\ \midrule
        \texttt{prior\_covariance} & \texttt{stan\_glmer}, \texttt{stan\_gamm4}, \texttt{stan\_nlmer}                     & Matrizes de covariância em modelos multiníveis                                                                           \\
        \bottomrule
    \end{tabular}
\end{frame}

\begin{frame}{\href{http://mc-stan.org/rstanarm/}{\texttt{rstanarm}} -- \textit{Prioris} Uniformes}
    Especifica-se colocando o valor \texttt{NULL} (nulo em \texttt{R}) nos argumentos
    \texttt{prior\_*} dos modelos \texttt{rstanarm}:
    \begin{vfilleditems}
        \item \lstinline!prior_intercept = NULL! -- \textbf{constante} possuirá \textit{priori} uniforme sobre todos os números reais $(-\infty, +\infty)$
        \item \lstinline!prior = NULL! -- \textbf{parâmetros} possuirão \textit{prioris} uniformes sobre todo os números reais $(-\infty, +\infty)$
        \item \lstinline!prior_aux = NULL! -- \textbf{parâmetros auxiliares} (geralmente o erro do modelo) possuirão \textit{prioris} uniforme sobre todos os números reais $(-\infty, +\infty)$\footnote{No caso de erro do modelo, isto se restringe aos numeros reais positivos: $[0, +\infty)$}.
    \end{vfilleditems}
\end{frame}

\begin{frame}[fragile]{\href{http://mc-stan.org/rstanarm/}{\texttt{rstanarm}} -- \textit{Prioris} Uniformes}
    \begin{lstlisting}
    stan_glm(y ~ ...,
        @prior = NULL@,
        @prior_intercept = NULL@,
        @prior_aux = NULL@)
    \end{lstlisting}
\end{frame}

\begin{frame}{\href{http://mc-stan.org/rstanarm/}{\texttt{rstanarm}} -- \textit{Prioris} Informativas}
    Coloca-se qualquer distribuição nos argumentos
    \lstinline!prior_*! dos modelos \texttt{rstanarm}:
    \begin{vfilleditems}
        \item \lstinline!prior = normal(0, 5)! -- $\text{Normal}(0, 5)$
        \item \lstinline!prior_intercept = student_t(4, 0, 10)! -- $\text{Student}(\nu=4, 0, 10)$
        \item  \lstinline!prior_aux = cauchy(0, 3)! -- $\text{Cauchy}^+(0, 3)$\footnote{Note que como ela é especificada como uma \textit{priori} de um parâmetro auxiliar (no caso o erro do modelo), ela somente tomará valores positivos por isso o $\text{Cauchy}^+$}
    \end{vfilleditems}
\end{frame}

\begin{frame}[fragile]{\href{http://mc-stan.org/rstanarm/}{\texttt{rstanarm}} -- \textit{Prioris} Informativas}
    \begin{lstlisting}
    stan_glm(y ~ ...,
        @prior = normal(0, 5)@,
        @prior_intercept = student_t(4, 0, 10)@,
        @prior_aux = cauchy(0, 3)@
        )
    \end{lstlisting}
\end{frame}

\begin{frame}{\href{http://mc-stan.org/rstanarm/}{\texttt{rstanarm}} -- \textit{Prioris} Padrões em Modelos Gaussianos}
    \begin{vfilleditems}
        \item \textbf{Constante} (\textit{Intercept}): Normal centralizada com média $\mu_{\boldsymbol{y}}$ e desvio padrão de $2.5 \sigma_{\boldsymbol{y}}$ - \lstinline!prior_intercept = normal(mean_y, 2.5 * sd_y)!
        \item \textbf{Coeficientes}: Normal para cada coeficiente média $\mu = 0$ e desvio padrão de $2.5\times\frac{\sigma_{\boldsymbol{y}}}{\sigma_{\boldsymbol{x}}}$ - \lstinline!prior = normal(0, 2.5 * sd_y/sd_xk)!
    \end{vfilleditems}
Em notação matemática:
$$
\begin{aligned}
\alpha &\sim \text{Normal}(\mu_{\boldsymbol{y}}, 2.5 \cdot \sigma_{\boldsymbol{y}})\\
\boldsymbol{\beta} &\sim \text{Normal}\left( 0, 2.5 \cdot \frac{\sigma_{\boldsymbol{y}}}{\sigma_{\boldsymbol{x}}} \right) \\
\sigma &\sim \text{Exponential}\left( \frac{1}{\sigma_{\boldsymbol{y}}} \right)
\end{aligned}
$$
\end{frame}

\begin{frame}{\href{http://mc-stan.org/rstanarm/}{\texttt{rstanarm}} -- \textit{Prioris} Padrões em Modelos Lineares Generalizados}
    \begin{vfilleditems}
        \item \textbf{Constante} (\textit{Intercept}): Normal centralizada com média $\mu = 0$ e desvio padrão de $2.5 \sigma_{\boldsymbol{y}}$ - \lstinline!prior_intercept = normal(0, 2.5 * sd_y)!
        \item \textbf{Coeficientes}: Normal para cada coeficiente média $\mu = 0$ e desvio padrão de $2.5\times\frac{1}{\sigma_{\boldsymbol{x}}}$ - \lstinline!prior = normal(0, 2.5 * 1/sd_xk)!
    \end{vfilleditems}
Em notação matemática:
$$
\begin{aligned}
\alpha &\sim \text{Normal}(0, 2.5 \cdot \sigma_{\boldsymbol{y}})\\
\boldsymbol{\beta} &\sim \text{Normal}\left( 0, 2.5 \cdot \frac{1}{\sigma_{\boldsymbol{x}}} \right) \\
\sigma &\sim \text{Exponential}\left( \frac{1}{\sigma_{\boldsymbol{y}}} \right)
\end{aligned}
$$
\end{frame}

\subsubsection{\textit{Prioris} do \texttt{brms}}
\begin{frame}{\textit{Prioris} do \texttt{brms}}
    \texttt{rstanarm} faz com que as \textit{prioris} sejam automaticamente
    transformadas para possuírem média $0$ e desvio padrão/variância $1$. O resultado
    disso é que a \textbf{escala das \textit{prioris} são unitárias}.
    \vfill
    \href{https://paul-buerkner.github.io/brms/}{\texttt{brms}} \textbf{não faz isso}.
    Por padrão, as \textit{prioris} não são transformadas, portanto, \textbf{se
    atente com a escala as \textit{prioris} do \texttt{brms}}.
\end{frame}

\begin{frame}[fragile]{\href{https://paul-buerkner.github.io/brms/}{\texttt{brms}} -- \textit{Prioris} Informativas}
    \begin{lstlisting}[basicstyle=\small]
brm(y ~ ...,
  prior = c(
    @set_prior("normal(0, 5)", class = "b", coef = "...")@,
        ...
    @set_prior("student_t(4, 0, 10)", class = "b", coef = "intercept")@,
    @set_prior("cauchy(0, 3)", class = "sigma")@
    )
  )
    \end{lstlisting}
\end{frame}

\begin{frame}{\href{https://paul-buerkner.github.io/brms/}{\texttt{brms}} -- \textit{Prioris} Padrões}
    Como o \href{https://paul-buerkner.github.io/brms/}{\texttt{brms}} dá
    muito mais autonomia, poder e flexibilidade ao usuário, todas as \textit{prioris}
    padrões dos coeficientes dos modelos são literalmente \textit{prioris} uniformes
    sobre todo os números reais $(-\infty, +\infty)$. \texttt{brms} apenas ajusta de
    maneira robusta a \textit{priori} para:
    \begin{vfilleditems}
        \item Constante (\textit{Intercept}): $t$ de Student média $\mu = \text{mediana}(y)$, desvio padrão de $\text{MAD}(y)$ e graus de liberdade $\nu = 3$
        \item Erro (\texttt{sigma}): $t$ de Student média $\mu = 0$ e desvio padrão de $\text{MAD}(y)$\footnote{\textit{\textbf{M}ean \textbf{A}bsolute \textbf{D}eviation} -- Desvio Absoluto Médio}
    \end{vfilleditems}
Em notação matemática:
$$
\begin{aligned}
\alpha &\sim \text{Student}(\nu=3, \text{mediana}({\boldsymbol{y}}), \text{MAD}({\boldsymbol{y}}))\\
\boldsymbol{\beta} &\sim \text{Unif}(-\infty, + \infty) \\
\sigma &\sim \text{Student}(\nu=3, 0, \text{MAD}({\boldsymbol{y}}))
\end{aligned}
$$
\end{frame}
